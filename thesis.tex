\documentclass[12pt]{book}
\usepackage[utf8]{inputenc}
\usepackage{indentfirst}
\usepackage{verbatim}
\usepackage{amssymb} % Mathe
\usepackage{amsmath} % Mathe
\pagestyle{headings}

\setlength{\paperwidth}{21cm}
\setlength{\paperheight}{29.5cm}
\setlength{\parindent}{6mm}
\setlength{\hoffset}{-1in}
\setlength{\voffset}{-1in}
\setlength{\oddsidemargin}{2.7cm}
\setlength{\evensidemargin}{2.2cm}
\setlength{\textwidth}{15.5cm}
\setlength{\textheight}{22.8cm}
\setlength{\topmargin}{2.5cm}
\setlength{\headheight}{0.5cm}
\setlength{\headsep}{1cm}
\setlength{\footskip}{1.4cm}


\begin{document}

\frontmatter

\title{Titel der Arbeit}

\author{Namen des Autors}

\date{\vspace{10mm} Diplomarbeit\\ \vspace{3mm} \today \\
\vspace{20mm} Thema gestellt von\\ \vspace{3mm} Prof. Dr.
W.-M. Lippe\\ \vspace{3mm} WWU - Institut f"ur Informatik}

\maketitle


\rule{0mm}{1mm}
\newpage
\rule{0mm}{1mm}
\newpage

F"urs Pr"ufungsamt ein paar Seiten frei.

\tableofcontents


\chapter{Einleitung}

Hier die Einleitung, Motivation schreiben.


\mainmatter

Einzelne Kapitel werden eingeladen:

\input{introduction}

\newpage

\backmatter

\appendix

\chapter{Fazit und Ausblick}

Wie der Name schon sagt...


\chapter{Anleitung zum Programm}

Eine Anleitung darf nicht fehlen.

\chapter{Programmlisting}

\begin{tiny}

Kleine Schrift spart Platz.\\

Mit verbatiminput bleibt das Layout eines Listings erhalten.\\

\verbatiminput{progamm}

\end{tiny}


\begin{thebibliography}{90}

\bibitem{Autor}
Autor\\
Titel\\
Verlag\\

\end{thebibliography}


\chapter{Erkl"arung}

\setlength{\parindent}{0mm}
Hiermit versichere ich, da"s ich die vorliegende Arbeit
selbständig verfa"st und keine anderen als die angegebenen
Hilfsmittel verwendet habe.\\

\vspace{20mm}

M"unster, \today\\

\vspace{15mm}

Name\\


\end{document}

