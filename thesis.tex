\documentclass[12pt]{book}
\usepackage[utf8]{inputenc}
\usepackage{indentfirst}
\usepackage{verbatim}
\usepackage{amssymb} % Mathe
\usepackage{amsmath} % Mathe
\pagestyle{headings}

\setlength{\paperwidth}{21cm}
\setlength{\paperheight}{29.5cm}
\setlength{\parindent}{6mm}
\setlength{\hoffset}{-1in}
\setlength{\voffset}{-1in}
\setlength{\oddsidemargin}{2.7cm}
\setlength{\evensidemargin}{2.2cm}
\setlength{\textwidth}{15.5cm}
\setlength{\textheight}{22.8cm}
\setlength{\topmargin}{2.5cm}
\setlength{\headheight}{0.5cm}
\setlength{\headsep}{1cm}
\setlength{\footskip}{1.4cm}


\begin{document}

\frontmatter

\title{Detexify: Hand-drawn \LaTeX-symbol recognition}

\author{Daniel Kirsch}

\date{\vspace{10mm} Diploma thesis\\ \vspace{3mm} \today \\
\vspace{20mm}
% Thema gestellt von\\ \vspace{3mm} Prof. Dr. W.-M. Lippe\\
\vspace{3mm} University of Münster - Department of Computer Science}

\maketitle


\rule{0mm}{1mm}
\newpage
\rule{0mm}{1mm}
\newpage

F"urs Pr"ufungsamt ein paar Seiten frei.

\tableofcontents

%!TEX root = /Users/daniel/Documents/thesis/thesis.tex
\chapter{Motivation}

\section{\LaTeX}

Wenn es um das Verfassen wissenschaftlicher Texte geht, ist für viele \LaTeX\ die erste Wahl. Anders als bei einem \acs{WYSIWYG}-Editor kümmert man sich nicht um das Layout, um Abstände und Schriftgrößen sondern um die Semantik des Geschriebenen. Stattdessen zeichnet man die Kapitel, Abschnitte, Formeln usw. semantisch aus, eher wie in \acs{HTML}. Der Vorteil liegt klar auf der Hand. Inhalt und Präsentation sind klar getrennt, was zur Folge hat, dass man nicht vom Wesentlichen abgelenkt wird, wenn man an Texten und Formeln arbeitet, und dass das Aussehen des Dokuments zentral in der Präambel definiert wird.

\LaTeX\ hat jedoch auch seine Nachteile. Gerade Anfänger haben es aufgrund der flachen Lernkurve schwer. Da man \LaTeX\ in der Regel im Quelltext bearbeitet, als jeden Befehl von Hand schreibt. Muss man sich unheimlich viel merken. Das beginnt bei einfachen Befehlen wie \texttt{\textbackslash chapter}, deren Name sich geradezu aufdrängt, aber wenn es darum geht Mathematische Formeln wie $$\Gamma \left( x \right) = \int\limits_0^\infty  {s^{x - 1} e^{ - s} ds}$$ in das Dokument zu bringen, wird es schon schwieriger.

Dass man ein $\Gamma$ mit dem Befehl \texttt{\textbackslash Gamma} bekommt ist noch zu erahnen. Um $\infty$ zu bekommen hätte man auch \texttt{\textbackslash infinity} statt \texttt{\textbackslash infty} versuchen können und hätte bloß eine Fehler geerntet, aber bei Befehlen wie \texttt{\textbackslash leftrightsquigarrow} ($\leftrightsquigarrow$) hört jede Intuition auf. Es ist also offensichtlich, dass es einigen Raum für unterstützende Maßnahmen bei der Erstellung von \LaTeX-Dokumenten gibt. Das gilt insbesondere für mathematische Formeln in denen viele unterschiedliche Symbole vorkommen können.

\section{Die optimale Eingabemethode}

Überlegt man sich die natürlichste Art Text oder Mathematik zu notieren, so kommt man unweigerlich auf Stift und Papier. Während bei Texten die Eingabe über eine Computertastatur durchaus schneller sein kann, als das Schreiben mit einem Stift (TODO Belege!) ist spätestens bei Formeln klar, dass hier der Stift klar im Vorteil ist. Es können beliebige Formen, Zeichen und Symbole in beliebige räumliche Beziehung gebracht werden. Nichts liegt also näher, als diese Eingabeform in die digitale Welt übertragen zu wollen.

Als Eingabegerät bietet sich also ein Grafiktablett an. Ein entsprechender \LaTeX-Editor würde eine Fläche zur Verfügung stellen, auf die einfach geschrieben und skizziert wird. Die Kurven und Linien würden dann vom Editor in Text, Tabellen, Formeln und Diagramme überführt -- alles genau wie vom Benutzer erwartet. Leider sind wir noch nicht so weit.

Alleine der Bereich der mathematischen Formelerkennung ist noch nicht auf einem Level, auf dem man die verfügbaren Lösungen als benutzbar bezeichnen könnte. Ein Beispiel ist die kostenlose Software \href{http://www.inftyproject.org}{InftyEditor}. Einfache Formeln werden noch recht sicher erkannt, aber sobald die Komplexität der Formel steigt und vor allem sobald man Symbole braucht, die die Software gar nicht kennt, wird die Benutzung zum Frusterlebnis. Es wundert daher nicht, dass an diesem Problem viel aktuelle Forschung betrieben wird.

\TODO{Beispiele oder Zitate/Referenzen}

Das Problem teilt sich dabei in drei Teile auf. Das erste ist die Segmentierung der mathematischen Formel, bei der einzelne Symbole isoliert werden müssen. Als zweites müssen die einzelnen Symbole richtig erkannt werden. Schließlich müssen die Symbole über ihre räumliche Position in eine logische Beziehung zueinander gebracht werden. Diese in diesen Schritten gewonnenen Erkenntnisse können natürlich die Entscheidung in den jeweils anderen beeinflussen, indem man die Semantik einer Interpretation in der Erkennung mit einfließen lässt. \TODO Zitate? Das Problem ist also sehr komplex.

\TODO{Hier könnte ein wenig Forschungsübersicht gegeben werden}

\section{Suche nach Alternativen}

Die optimale Eingabemethode ist also (noch) nicht praktikabel. Um ganze Formeln zu erkennen haben wir noch nicht die optimalen Algorithmen gefunden. Eine Alternative ist, dem Computer nur einen Teil der Erkennung zu übertragen. Ein Beispiel hierfür ist \href{http://jequation.sourceforge.net/}{JEquation}. In diesem Programm wird dem Benutzer vorgegeben, wo der zu malen hat, damit die Struktur der Formel richtig erkannt wird. \TODO Bild? Es bleibt also die Aufgabe die einzelnen Symbole zu erkennen.



\href{http://lyx.org}{Lyx}


\section{Ein pragmatischer Ansatz}



\url{http://url.de}

\mainmatter
%!TEX root = /Users/daniel/Documents/thesis/thesis.tex
\chapter{Chapter}

\section{Section}

Lorem ipsum dolor sit amet, consectetur adipisicing elit, sed do eiusmod tempor incididunt ut labore et dolore magna aliqua. Ut enim ad minim veniam, quis nostrud exercitation ullamco laboris nisi ut aliquip ex ea commodo consequat. Duis aute irure dolor in reprehenderit in voluptate velit esse cillum dolore eu fugiat nulla pariatur. Excepteur sint occaecat cupidatat non proident, sunt in culpa qui officia deserunt mollit anim id est laborum.

\newpage

\backmatter

\appendix

%!TEX root = /Users/daniel/Documents/thesis/thesis.tex
\chapter{Zusammenfassung und Ausblick}

\TODO

\begin{itemize}
  \item Vergleich verschiedener innerer Maße
  \item Kombination von Klassifizierern
  \item Version, die ohne Web auskommt
  \item \dots
\end{itemize}

\chapter{Anleitung zum Programm}

Eine Anleitung darf nicht fehlen.

\chapter{Programmlisting}

\begin{tiny}

% Mit verbatiminput bleibt das Layout eines Listings erhalten.\\

\verbatiminput{progam.hs}

\end{tiny}

\begin{thebibliography}{90}

\bibitem{Autor}
Autor\\
Titel\\
Verlag\\

\end{thebibliography}

\chapter{Statement of authorship}

\setlength{\parindent}{0mm}
I hereby certify that this diploma thesis has been composed by myself, and describes my own work, unless otherwise acknowledged in the text. All references and verbatim extracts have been quoted, and all sources of information have been specifically acknowledged. It has not been accepted in any previous application for a degree.\\

\vspace{20mm}

Münster, \today\\

\vspace{15mm}

Daniel Kirsch\\

\end{document}

