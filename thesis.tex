%!TEX TS-program = xelatex
%!TEX encoding = UTF-8 Unicode
\documentclass[12pt]{book}
\synctex=1
\usepackage{amssymb} % Mathe
\usepackage{amsmath} % Mathe
\usepackage[usenames,dvipsnames]{color}
\usepackage{listings}
\lstdefinelanguage{Javascript} {
	morekeywords={
		break,const,continue,delete,do,while,export,for,in,function,
		if,else,import,in,instanceOf,label,let,new,return,switch,this,
		throw,try,catch,typeof,var,void,with,yield
	},
	sensitive=false,
	morecomment=[l]{//},
	morecomment=[s]{/*}{*/},
	morestring=[b]",
	morestring=[d]'
}
\lstset{
  language=Javascript,
	frame=tb,
	framesep=5pt,
	basicstyle=\footnotesize\ttfamily,
	showstringspaces=false,
	keywordstyle=\ttfamily\bfseries\color{CadetBlue},
	identifierstyle=\ttfamily,
	stringstyle=\ttfamily\color{OliveGreen},
	commentstyle=\color{GrayBlue},
	rulecolor=\color{Gray},
	xleftmargin=5pt,
	xrightmargin=5pt,
	aboveskip=\bigskipamount,
	belowskip=\bigskipamount
}

% Math Fonts
\usepackage[mathcal]{euler}
\usepackage[cm-default]{fontspec}
% FONTS
\usepackage{xunicode}
\usepackage{xltxtra}
\defaultfontfeatures{Mapping=tex-text} % converts LaTeX specials (``quotes'' --- dashes etc.) to unicode
\setromanfont [Ligatures={Common}, Numbers={OldStyle}]{Adobe Caslon Pro}
\setmonofont[Scale=0.8]{Monaco} 
\setsansfont[Scale=0.9]{Optima Regular} 
% ---- CUSTOM AMPERSAND
\newcommand{\amper}{{\fontspec[Scale=.95]{Adobe Caslon Pro}\selectfont\itshape\&}}

\newcommand{\TODO}{\textsc{\textbf{\textcolor{red}{Todo\ }}}}
\newcommand{\FIXME}{\textsc{\textbf{\textcolor{red}{Fixme\ }}}}

% HEADINGS
\pagestyle{headings}
% \usepackage{sectsty} 
% \usepackage[normalem]{ulem} 
% \sectionfont{\sffamily\mdseries\large} 
% \subsectionfont{\rmfamily\mdseries\scshape\normalsize} 
% \subsubsectionfont{\rmfamily\bfseries\upshape\normalsize} 

% DOCUMENT LAYOUT
\setlength{\paperwidth}{21cm}
\setlength{\paperheight}{29.5cm}
\setlength{\parindent}{6mm}
\setlength{\hoffset}{-1in}
\setlength{\voffset}{-1in}
\setlength{\oddsidemargin}{2.7cm}
\setlength{\evensidemargin}{2.2cm}
\setlength{\textwidth}{15.5cm}
\setlength{\textheight}{22.8cm}
\setlength{\topmargin}{2.5cm}
\setlength{\headheight}{0.5cm}
\setlength{\headsep}{1cm}
\setlength{\footskip}{1.4cm}

% PDF SETUP
% ---- FILL IN HERE THE DOC TITLE AND AUTHOR
\usepackage[xetex, bookmarks, colorlinks, breaklinks, pdftitle={LaTeX-Erkennung},pdfauthor={Daniel Kirsch}]{hyperref}  
\hypersetup{linkcolor=blue,citecolor=blue,filecolor=black,urlcolor=blue} 

% Stuff
\usepackage[ngerman]{babel}
\usepackage{indentfirst}
\usepackage[printonlyused]{acronym}
\usepackage[numbers,round]{natbib}

% \pagestyle{headings}
% 

\begin{document}  

\frontmatter

\title{Detexify: Erkennung handgemalter \LaTeX-Symbole \\
\vspace{5mm} \includegraphics[height=7.5cm]{figures/icon.png} }

\author{Daniel Kirsch}

\date{\vspace{10mm} Diplomarbeit\\ \vspace{3mm} \today \\
\vspace{20mm}
% Thema gestellt von\\ \vspace{3mm} Prof. Dr. W.-M. Lippe\\
\vspace{3mm} Westfälische Wilhelms-Universität Münster\\Institut für Informatik}

\maketitle

% \newpage\mbox{}\vfill
% \begin{center}
% \large
% \includegraphics{figures/icon.png}
% \normalsize
% \end{center}
% \vfill\mbox{}\newpage

\rule{0mm}{1mm}
\newpage
\rule{0mm}{1mm}
\newpage

% Fürs Prüfungsamt ein paar Seiten frei.

\listoffigures    % Abbildungsverzeichnis
\listoftables     % Tabellenverzeichnis
\tableofcontents  % Inhaltsverzeichnis

%!TEX root = /Users/daniel/Documents/thesis/thesis.tex
\chapter{Vorwort}

\section*{Wie es zu dieser Diplomarbeit kam...}

Eines Tages saß ich mit einem Kommilitonen zusammen in der Mensa. Wir unterhielten uns über Ideen im Allgemeinen und welche es sich lohne umzusetzen. Wir hatten viele Ideen und meistens keine Zeit sie zu verwirklichen. Tatsächlich existierte sogar eine Liste mit einigen davon - ein schlichtes Blatt Papier, auf dem sich über die Zeit einiges angesammelt hatte und wir nannten es den Ideenfriedhof.

Am besagten Tag ging es vor allem darum, wie man auf {\em gute} Ideen käme. Denn auf dem Ideenfriedhof ruhte Sinnvolles und weniger Sinnvolles. Ich äußerte die Behauptung, dass Ideen, die eigene Probleme lösen, solche seien, deren Umsetzung sich lohne, weil man in der Regel mit seinen Problemen nicht alleine dastehe. Somit löse man auch Probleme für andere.

Wir kamen auf andere Themen und da mein Kommilitone zu jener Zeit den Sekretärinnen des Mathematischen Instituts einen \LaTeX-Kurs gab, unterhielten wir uns auch darüber. Irgendwann bemerkte er: "`Da fällt mir etwas ein... Was fehlt ist eine \LaTeX-Rückwärts-Suche. Es kommt ganz häufig vor, dass man zwar weiß, was man für ein Symbol haben will, aber nicht den Befehl kennt. Das Problem hatten meine Sekretärinnen jetzt auch schon ein paar Mal. Wenn man jetzt ein Programm hätte, in das man das Symbol malen könnte und das dann den richtigen Befehl ausspuckt..."'

Ich fand, dass das eine außergewöhnlich gute Idee war. Ich hatte das Problem zwar nicht selbst, aber ich würde es bald haben, wenn ich meine Diplomarbeit schreiben würde, denn ich würde sie in \LaTeX\ schreiben. Ich hatte allerdings zu dem Zeitpunkt noch keine Ahnung, was ich als Diplomarbeit schreiben würde. Ich hatte zu diesem Zeitpunkt auch keine Berührung mit Mustererkennung irgendeiner Art gehabt und keine Ahnung von \LaTeX. Ich fand aber das Problem so interessant, dass ich fest entschlossen war es anzugehen.

Einige Wochen später, nach vielem Lesen und Probieren, hatte ich eine benutzbare Version, implementiert als Website und bedienbar über einen Browser. Seitdem wird die Anwendung täglich von fast 1000 Besuchern genutzt. Erst später wurde mir klar, dass sich das Thema gut für eine Diplomarbeit eignet und zu meiner Freude war Prof. Jiang auch dieser Meinung. Zudem würde ich meine eigene Diplomarbeit bei der Erstellung meiner Diplomarbeit verwenden können. Wer kann das schon von sich behaupten?

Was aber ich eigentlich hier sagen will: Sie lesen nun im wesentlichen das Ergebnis eines \LaTeX-Kurses für Sekretärinnen des Mathematischen Institutes der WWU Münster. Darum gilt den damaligen Teilnehmerinnen mein besonderer Dank.

\newpage\mbox{}\vfill
\begin{center}
\large
Für Julia Fiege, Yu-Mei Kao, Angela Odermann \amper~ Tamara Tietmeyer.
\normalsize
\end{center}
\vfill\mbox{}\newpage

\mainmatter
%!TEX root = /Users/daniel/Documents/thesis/thesis.tex
\chapter{Motivation}

\section{\LaTeX}

Wenn es um das Verfassen wissenschaftlicher Texte geht, ist für viele \LaTeX\ die erste Wahl. Anders als bei einem \ac{WYSIWYG}-Editor kümmert man sich nicht um das Layout, um Abstände und Schriftgrößen sondern um die Semantik des Geschriebenen. Man zeichnet die Kapitel, Abschnitte, Formeln usw. aus, eher wie in \ac{HTML} als in MS-Word. Der Vorteil liegt auf der Hand. Inhalt und Präsentation sind klar getrennt, was zur Folge hat, dass man nicht vom Wesentlichen abgelenkt wird, wenn man an Texten und Formeln arbeitet, und dass das Aussehen des Dokuments zentral in der Präambel definiert wird.

\LaTeX~hat jedoch auch seine Nachteile. Gerade Anfänger haben es aufgrund der flachen Lernkurve schwer. Da man \LaTeX~in der Regel im Quelltext bearbeitet, also jeden Befehl von Hand schreibt, muss man sich unheimlich viel merken. Das beginnt bei einfachen Befehlen wie \texttt{\textbackslash chapter}, deren Name sich geradezu aufdrängt, aber wenn es darum geht Mathematische Formeln wie $$\Gamma \left( x \right) = \int\limits_0^\infty  {s^{x - 1} e^{ - s} ds}$$ in das Dokument zu bringen, wird es schon schwieriger.

Dass man ein $\Gamma$ mit dem Befehl \texttt{\textbackslash Gamma} bekommt ist noch zu erahnen. Um $\infty$ zu bekommen hätte man auch \texttt{\textbackslash infinity} statt \texttt{\textbackslash infty} versuchen können und hätte bloß einen Fehler geerntet, aber bei Befehlen wie \texttt{\textbackslash leftrightsquigarrow} ($\leftrightsquigarrow$) hört jede Intuition auf. Es ist also offensichtlich, dass es einigen Raum für unterstützende Maßnahmen bei der Erstellung von \LaTeX-Dokumenten gibt. Das gilt insbesondere für mathematische Formeln in denen viele unterschiedliche Symbole vorkommen können.

\section{Die optimale Eingabemethode}

Überlegt man sich die natürlichste oder zumindest gewohnteste Art Text oder Mathematik zu notieren, so kommt man unweigerlich auf Stift und Papier. Während bei Texten die Eingabe über eine Computertastatur durchaus schneller sein kann als das Schreiben mit einem Stift (\TODO Belege! Da hab ich doch irgendwas gelesen?!? APTE A., KIMURA T.: A comparison study of the pen and the mouse in editing graphic diagrams. In Proceedings of the IEEE Symposium on Visual Languages (1993), pp. 352–357. nicht so ganz :/), ist spätestens bei Formeln klar, dass hier der Stift im Vorteil ist. Es können beliebige Formen, Zeichen und Symbole in beliebige räumliche Beziehung gebracht werden. Nichts liegt also näher, als diese Eingabeform in die digitale Welt übertragen zu wollen.

Als Eingabegerät bietet sich also ein Grafiktablett an. Ein entsprechender \LaTeX-Editor würde eine Fläche zur Verfügung stellen, auf die einfach geschrieben und skizziert wird. Die Kurven und Linien würden dann vom Editor in Text, Tabellen, Formeln und Diagramme überführt -- alles genau wie vom Benutzer erwartet. Leider sind wir noch nicht so weit.

Alleine der Bereich der mathematischen Formelerkennung ist noch nicht auf einem Level, auf dem man die verfügbaren Lösungen als gut bezeichnen könnte. Es gibt kaum kommerzielle Lösungen und eine Hand voll experimentelle Tools. Nur wenige davon haben überhaupt \LaTeX~oder Textsatz allgemein als Fokus. Andere gehen eher in Richtung interaktive Mathematik oder sogar \ac{CAS}.

\subsection{Infty Editor}

Ein Beispiel für einen Editor, der beim Textsatz helfen soll ist die kostenlose Software \href{http://www.inftyproject.org}{InftyEditor} \cite{Suzuki:2003p786}. Sie ermöglicht die Eingabe von Formeln in einem interessanten Mix aus \ac{WYSIWYG}-Editor und \LaTeX-Autovervollständigung mit Vorschau (siehe Abb.~\ref{fig:inftyeditor-autocomplete}). 

InftyEditor hat außerdem eine Funktion handgeschriebene Formeln zu erkennen. Dazu öffnet man das InftyHandWriting Input Pad und fängt an zu zeichnen. Einfache Formeln werden damit auch noch recht sicher erkannt, aber sobald die Komplexität der Formel steigt und vor allem sobald man Symbole braucht, die die Software gar nicht kennt, wird die Benutzung zum Frusterlebnis. Abb.~\ref{fig:inftyeditor} zeigt das InftyHandwriting Input Pad.

InftyEditor ist kostenlos aber die Software InftyReader vom gleichen Hersteller, eine Mathematik-OCR-Software, ist ein kommerzielles Produkt. \TODO Link?

\begin{figure}
  \begin{center}
    \includegraphics[width=\textwidth]{figures/inftyeditor-autocomplete.png}
  \end{center}
  \caption{InftyEditor Autovervollständigung}
  \label{fig:inftyeditor-autocomplete}
\end{figure}

\begin{figure}
  \begin{center}
    \includegraphics[width=\textwidth]{figures/inftyeditor.png}
  \end{center}
  \caption{InftyEditor}
  \label{fig:inftyeditor}
\end{figure}


\subsection{Freehand Formula Entry System}

http://research.cs.queensu.ca/drl/ffes/

Lorem Ipsum\dots \TODO Paper ansehen

\subsection{Microsoft Math}

Microsoft Math ist ein kommerzielles Produkt, das aber nicht für den Textsatz gedacht ist, sondern eher mit einem programmierbaren Taschenrechner zu vergleichen ist. Die Eingabe erfolgt entweder über die Tatstatur (Abb. \ref{fig:ms-math-keyboard}) oder über Handschrifterkennung (Abb. \ref{fig:ms-math-ink}) und die Handschrifterkennung macht zwar einen solideren Eindruck als bei den vorgenannten Produkten, ist aber auch noch nicht der Weissheit letzter Schluss. Abb. \ref{fig:ms-math-ink} zeigt einen Versuch die Gammafunktion zu erkennen und der Erfolg ist zweifelhaft.

Ist die Formel dann auf welche Weise auch immer eingegeben, so kann dann damit einiges angestellt werden, was aber nicht im Interesse dieser Arbeit liegt und darum auch nicht näher beschrieben wird.

\begin{figure}
  \begin{center}
    \includegraphics[width=\textwidth]{figures/ms-math-keyboard.png}
  \end{center}
  \caption{Microsoft Math 3.0 Tastatureingabe}
  \label{fig:ms-math-keyboard}
\end{figure}

\begin{figure}
  \begin{center}
    \includegraphics[width=\textwidth]{figures/ms-math-ink.png}
  \end{center}
  \caption{Microsoft Math 3.0 Handschrifterkennung}
  \label{fig:ms-math-ink}
\end{figure}


\subsection{MathJournal}

Lorem Ipsum\dots \TODO Paper ansehen

\subsection{MathBrush}

Lorem Ipsum\dots \TODO Paper ansehen

\TODO Sollte noch erwähnen, dass ich nur LaTeX machen will und kein ComputerAlgebra..

\section{Was ist so schwierig?}

Das Problem teilt sich dabei in drei Teile auf. Das erste ist die Segmentierung der mathematischen Formel, bei der einzelne Symbole isoliert werden müssen. Als zweites müssen die einzelnen Symbole richtig erkannt werden. Schließlich müssen die Symbole über ihre räumliche Position in eine logische Beziehung zueinander gebracht werden. Die in diesen Schritten gewonnenen Erkenntnisse können natürlich die Entscheidung in den jeweils anderen beeinflussen, indem man die Semantik einer Interpretation in der Erkennung mit einfließen lässt. \TODO Zitate? Das Problem ist also sehr komplex.

Auf interaktive Fehlerbehebung eingehen? MathBrush hat sowas. Und ich hab sowas auch schon in den Artikeln gesehen


\section{Suche nach Alternativen}
\label{sec:alternativen}

\TODO Detexify ist trotzdem nützlich, selbst wenn man sowas wie Infty oder ffes benutzt um bei der Korrenktur einzende Symbole einzufügen die vorher mit Deteify gesucht werden können

Die optimale Eingabemethode ist also (noch) nicht praktikabel. Um ganze Formeln zuverlässig zu erkennen haben wir noch nicht die optimalen Algorithmen gefunden. Eine Alternative ist, dem Computer nur einen Teil der Erkennung zu übertragen. Ein Beispiel hierfür ist \href{http://jequation.sourceforge.net/}{JEquation}. In diesem Programm wird dem Benutzer vorgegeben, wo der zu malen hat, damit die Struktur der Formel richtig erkannt wird. Es bleibt also die Aufgabe die einzelnen Symbole zu erkennen. \TODO Bild? Und welche Kritik habe ich an diesem Ansatz?

Es gibt natürlich auch Ansätze, die ohne jede Form von Mustererkennung auskommen. Hier ist \href{http://lyx.org}{Lyx} ein Beispiel. Lyx ist ein \ac{WYSIWYM}-Editor, er funktioniert also ähnlich wie ein \ac{WYSIWYG}-Editor wie MS-Word. Dabei arbeitet man nicht direkt mit \LaTeX-Befehlen sondern hat einen graphischen Editor in dem man jedoch die Textabschnitte semantisch auszeichnet, statt den Schriftstil manuell vorzugeben. Für Mathematische Formeln enthält Lyx einen Formeleditor, der ähnlich dem funktioniert, was aus Office-Paketen wie MS-Word bekannt ist. Um Symbole einzufügen hat man einerseits die Möglichkeit direkt \LaTeX-Befehle (mit Auto-Vervollständigung) direkt einzugeben, oder das gesuchte Symbol aus mehreren Symboltabellen auszuwählen und per Klick einzufügen. Die erste der beiden Möglichkeiten setzt natürlich wieder voraus, dass der Author den Namen des Befehls kennt. Die zweite Methode hat den Nachteil, dass die Symboltabellen schnell unübersichtlich werden, wenn sie zu umfangreich sind. Es lässt sich also nur ein kleiner Teil der verfügbaren Symbole unterbringen, ohne den Nutzen zu kompromittieren.

\section{Ein pragmatischer Ansatz} % (fold)
\label{sec:pragmatisch}

Viele Benutzer fühlen sich aber durchaus wohl damit ihre Dokumente direkt im Quelltext zu bearbeiten. Texteditoren wie Vim, Emacs etc. erfreuen sich großer Beliebtheit zum erstellen von Dokumenten. (\TODO Wie belege ich das? Internet?) Ich selbst ziehe den Texteditor Vim trotz umfangreicher Recherchen in diesem Bereich jedem spezialisierten \LaTeX-Editor vor. Die Frage ist also, wie ein pragmatisches Werkzeug aussieht, dass einem typischen Anwender die Arbeit an seinem Dokument insbesondere die Eingabe von mathematischen Formeln erleichtert ohne dessen Arbeitsweise komplett umzukrempeln.

Ein Problem, das jeder Anwender egal ob Einsteiger oder Fortgeschrittener einmal hat, ist, dass er den Befehl für ein Symbol nicht weiss. Vielleicht hat er ihn noch nie gebraucht, oder er ist ihm entfallen. Das Symbol ist auch nicht in den Symboltabellen in seinem \LaTeX-Editor (falls er nicht sowieso einen Texteditor verwendet). Also muss er ein Buch oder die "`Comprehensive \LaTeX\ Symbol List"'~\cite{Pakin:2009p2664} wälzen, um das gewünschte Symbol zu finden. Das geht bei der großen Anzahl an Symbolen~\footnote{\cite{Pakin:2009p2664} listet 4947 Symbole auf.} nicht besonders schnell. Genau hier könnte ein pragmatisches Werkzeug ansetzen und auf einem Mittelweg zwischen den in \ref{sec:alternativen} vorgestellten Werkzeugen eine Lösung für dieses Problem anbieten.

Das Werkzeug müsste also Symbolsuche bieten, durch die sehr schnell der Befehl zum gesuchten Symbol gefunden werden kann. Der Benutzer kann sein Dokument im Editor seiner Wahl bearbeiten und im Bedarfsfall das Suchwerkzeug aufrufen, um schnell wieder an die eigentliche Arbeit, das Verfassen des Dokuments, zu gehen. Um ein solches Werkzeug wird es im folgenden gehen.

% section ein_pragmatischer_ansatz (end)

%!TEX root = /Users/daniel/Documents/thesis/thesis.tex
\chapter[Detexify]{Detexify - \LaTeX-Symbolsuche als Webanwendung} % (fold)
\label{cha:detexify}

Bevor ich auf die technischen Details von Detexify eingehe sollten Sie, lieber Leser, sich kurz mit dem Programm vertraut gemacht haben. Die Beschreibungen sind dann viel verständlicher. Die Benutzung von Detexify sollte selbsterklärend sein. Sollten Sie doch Schwierigkeiten haben finden sie in \ref{cha:benutzerhandbuch} ein kurzes Benutzerhandbuch. Sie können auf Detexify mit einem modernen Browser\footnote[1]{Aktuelle Versionen von Firefox, Chrome, Safari und Opera funktionieren prima.} unter der Adresse \url{http://detexify.kirelabs.org} zugreifen.

\section{Die Architektur} % (fold)
\label{sec:architektur}

Bei jeder Anwendung muss man sich, bevor sie geschrieben wird, entscheiden, wie die Anwendung zur Verfügung gestellt werden soll. Daraus resultieren weitere Entscheidungen, wie z.B. die Wahl der Programmiersprache und der Datenformate.

Bei Detexify lag aus Augenmerk vor allem auf Plattformunabhängigkeit. Jeder sollte einfach und schnell Zugriff zum Programm haben, egal auf was für einem System er seine Dokumente bearbeitet und es sollte möglichst ohne Installation auskommen. Es sollte sich zudem einfach zentral warten lassen und vor allem sollten die Trainingsdaten für die Mustererkennung zentral gespeichert sein und außerdem leicht und jederzeit erweitern werden können. Um diese Anforderungen zu erfüllen wurde Detexify als Webanwendung implementiert.

\TODO Crowd-Sourced Training... Training einer so großen Datenbank für einen Einzelnen langwierig aber zu vielen...

\begin{itemize}
  \item \textbf{Plattformunabhängigkeit:} Heutzutage ist ein Webbrowser auf jedem Computer verfügbar. Um Detexify verwenden zu können wird nur ein moderner\footnotemark[1] Webbrowser benötigt.
  \item \textbf{Zentrale Wartung:} Fehlerbehebungen und Verbesserungen sind zentral anwendbar und stehen jedem Benutzer beim nächsten Besuch der Anwendung sofort zur Verfügung.
  \item \textbf{Zentrale Trainingsdaten:} Die Trainingsdaten sind zentral in einer Datenbank gespeichert. Jeder Benutzer profitiert 
\end{itemize}

% chapter detexify (end)
%!TEX root = /Users/daniel/Documents/thesis/thesis.tex
\chapter{Erkennung handgeschriebener Symbole} % (fold)
\label{cha:erkennung_handgeschriebener_symbole}

Die Erkennung handgeschriebener Symbole ist das Herzstück von Detexify. Hier wird einer unbekannten Eingabe von Daten ein \LaTeX-Symbol zugeordnet. Genau genommen wird der Eingabe eine Rangfolge von Symbolen zugeordnet, welche in der in \ref{sub:symbolsuche} beschriebenen Liste auftaucht, nachdem ein Nutzer ein Symbol gezeichnet hat.

Zur Erkennung einzelner handgeschriebener Symbole wurden bereits unterschiedlichste Klassifikationsverfahren verwendet \cite{Plamondon:2000p10303}. Um ein leistungsstarkes Backend für Detexify zu entwickeln musste also ein Verfahren ausgewählt und gegebenenfalls optimiert werden, dass den spezifischen Anforderungen der Anwendung und der Architektur gerecht wurde.

Die Anforderungen sind die folgenden:

\begin{description}
  \item[Adaptionsfähigkeit] Es sollte jeder Zeit ein Training zusätzlicher Symbole möglich sein.
  \item[Skalierbarkeit] Die Erkennungsraten sollten auch bei einer großen Anzahl von Klassen gut sein.
  \item[Laufzeitverhalten] Das Laufzeitverhalten sollte eine Erkennung in Echtzeit ermöglichen.
\end{description}

Ein besonderes Interesse galt in meinem Fall natürlichen den Verfahren, die für Symbolerkennung, insbesondere mathematischer Symbole, bereits erfolgreich eingesetzt wurden. Dabei handelt es sich um \ac{SVM}, \ac{HMM} und \ac{DTW}. Es gibt auch einige wenige Strukturelle Ansätze \cite{Genoe:2006p10601}.

Die Entscheidung ist im Fall von Detexify auf \ac{DTW} gefallen.

Eine Analyse zur Eignung der unterschiedlichen Verfahren findet sich in \ref{sec:klassifizierung}. Im folgenden wird zuerst auf die verwendete Terminologie und die Herausforderungen der Erkennung von \LaTeX-Symbolen eingegangen.

\section{Terminologie} % (fold)
\label{sec:terminologie}

Wie in jedem Forschungsgebiet hat sich auch bei der Handschrifterkennung eine Nomenklatur entwickelt, die die verschiendenen Aspekte des Themas beschreibt.

Man unterscheidet zwischen \emph{Online}- und \emph{Offline}-Systemen \cite{Tappert:1990p10302}. Bei Online-Systemen wird die Erkennung durchgeführt während der Nutzer schreibt. Die Striche werden dabei vom Eingabegerät (häufig ein Grafiktablett) als Funktion $ t \mapsto \alpha $, wobei $\alpha$ den Zustand der Stiftspitze kodiert, an das System übertragen. Das heißt, es stehen für die Erkennung alle dynamischen Eigenschaften des Geschriebenen zur Verfügung, wie die Anzahl, Reihenfolge und Richtung der einzelnen Pinselstriche. Im Gegensatz dazu verarbeiten Offline-Systeme Scans von Geschriebenem, arbeiten also zu einem Zeitpunkt, wenn der Schreibvorgang längst beendet ist. Sie können also außer Pixeln keine weiteren Informationen verwenden.
Detexify ist ein Online-System.

Der Zustand der Stiftspitze besteht in Online-Systemen in der Regel aus den Koordinaten $(x,y)$, der Information, ob die Spitze das Tablett berührt, oft bezeichnet mit \emph{pen-up} bzw. \emph{pen-down} und in manchen Fällen auch aus der Neigung und dem Azimut.

Aus den Daten werden dann häufig Eigenschaften abgeleitet, sog. \emph{Features}. Man unterscheidet zwischen \emph{globalen} und \emph{lokalen} Features \cite{Tapia:2007p9160}. Lokale Features sind solche, die von einzelnen Punkten auf einem Strich und dessen benachbarten Punkten abgeleitet werden. Globale Features werden hingegen von der Menge der Striche als Ganzes abgeleitet.

\section[Herausforderungen]{Herausforderungen der Erkennung von \LaTeX-Symbolen}
Die Erkennung von handgemalten \LaTeX-Symbolen ist allein durch die enorme Anzahl der Symbole eine große Herausforderung \cite{Koerich:2003p1562}. Hinzu kommt, dass die Symbole aus sehr unterschiedlichen Alphabeten kommen. Es kommen lateinische, griechische, mathematische und weitere Symbole vor. Im Gegensatz zu asiatischen Sprachen, die ebenfalls sehr große Alphabete aufweisen\footnote{In Japan werden heute 6000-7000 Buchstaben verwendet. In China ist die Anzahl der im täglichen Leben verwendeten Buchstaben bei etwa 5000 \cite{Jaeger:2003p1097}}, ist jedoch die Anzahl und Reihenfolge der Striche nicht vorgegeben, was die Erkennung zusätzlich verkompliziert \cite{Watt:2005p1816}. Außerdem gibt es sehr viele sehr ähnliche Symbole wie $\rightarrow,\mapsto,\leadsto,\rightharpoonup,\hookrightarrow,\rightarrowtail$. Dazu kommen noch die Probleme vor der jede Handschrifterkennung steht, wie unterschiedliche Schreibstile sowohl von unterschiedlichen Schreibern als auch natürliche Variationen in der Schreibweise eines Einzelnen.

Ein guter Klassifizierer muss also einiges leisten und der Situation gerecht zu werden.

\section{Preprocessing und Normalisierung} % (fold)
\label{sec:preprocessing_und_normalisierung}

Eine Vorverarbeitung der Daten bevor sie an die Erkennungsalgorithmen ausgeliefert werden, ist eine wirksame Methode um Rauschen, das z.B. durch Ungenauigkeiten der Eingabegeräte oder Unachtsamkeit des Schreibers, entstehen kann, zu relativieren. Vorverarbeitung kann aber auch überflüssige Informationen entfernen, die vom verwendeten Erkennungsalgorithmus nicht verwendet werden. Zudem kann durch Normalisierung der Daten ungewollte Variation reduziert werden. Es gibt kaum ein System zu Handschrifterkennung, das keine Vorverarbeitung durchführt.

Wie gesagt verwendet Detexify zur Klassifizierung \ac{DTW}, was natürlich das Preprocessing beeinflusst. In Detexify werden die folgenden Maßnahmen ergriffen:

\begin{description}
  \item[Normalisierung der Größe und Position]
    Die ankommenden Striche werden verschoben und skaliert, so dass sie unter Beibehaltung ihres Seitenverhältnisses zentriert und maximal im Quadrat $[0,1]\times[0,1]$ liegen. Das ist notwendig, damit \ac{DTW} sinnvoll angewandt werden kann.
  %\item[Entfernung von Haken]
    %An den Enden von Strichen kann es zu Haken kommen, die beim Aufsetzen\dots
  \item[Glättung]
    Jeder Punkt eines Striches wird durch das arithmetische Mittel des Punktes und seiner beiden Nachbarn ersetzt.
  \item[Äquidistante Verteilung]
    Da die Zeichenfläche im Browser in Detexify eine gewisse Abtastrate hat, die erstens Herstellerabhängig sein kann, und zweitens zeitabhängig ist, kann die Verteilung der Punkte auf den Strichen sehr ungleich sein. Daher werden die Punkte neu verteilt, so dass sie Entfernung zwischen zwei je Punkten gleichmäßig groß ist. Die Anfangs- und Endpunkte von Strichen werden dabei erhalten.
  \item[Verkettung]
    Da \ac{DTW} zwei Zeitreihen vergleicht, werden die Striche miteinander verkettet, so dass das Abstandsmaß direkt angewendet werden kann. \TODO das ist doch Mist! Das muss ich begründen. Verschlechtert ja die Erkennung total.
\end{description}

% section preprocessing_und_normalisierung (end)

\section{Klassifizierung} % (fold)
\label{sec:klassifizierung}

Die Klassifizierung der normalisierten Daten 

\cite{Tappert:1990p10302} sagt bei Online-Erkennung reichen wegen der Interaktivität einfachere Methoden, was bei mir ja voll zutrifft.

\begin{itemize}
  \item KNN - bevorzugt in Japan \cite{Jaeger:2003p1097} / auch template matching genannt / Abstandsmaß entscheidend
  \item SVM - binärer klassifizierer ~> nicht so sinnvoll bei vielen klassen, bes. lineare SVMs nicht sinnvoll weil keine konvexen Klassen
  \item HMM - substrokemodelling macht keinen sinn + brauchen keine segmentierung
  \item Nets - ganz schlecht bei vielen Klassen laut \cite{Jaeger:2003p1097}
\end{itemize}
- SVM z.B. JMathNotes (MathFor) \TODO Paper suchen! siehe \cite{Vuong:2010p10279} - auch SVM \cite{Golubitsky:2009p2456}
- KNN z.B. FFES mit 50 features und verm. euklidischem Abstand \TODO Paper suchen! siehe \cite{Vuong:2010p10279}

Natural Log, symbol recognition is implemented using a statistical approach based on Gaussian Density Estimation \dots siehe auch \cite{Vuong:2010p10279} \TODO Paper suchen

Among the distance measures used for classifying handwritten mathematical sym- bols, the elastic matching distance is known to be one of the most accurate \cite{Golubitsky:2009p2433}
 
\cite{Vuong:2010p10279} benutzt Elastic Matching mit Steigung und Krümmung

MathBrush \cite{Labahn:2008p10301} benutzt Kombi aus Elastic Matching und anderen Methoden siehe auch \cite{MacLean:2010p9970} DTW in linear time and constant space

Strukturelle Methoden gehen auch nicht gut, weil unterschiedliche Alphabete ~> keine gemeinsame Struktur

\subsection[DTW]{Dynamic Time Warping}
\label{sub:dtw}

- DTW/Elastic matching/DP matching wird benutzt für Lateinische Schriftzeihen, Chinesisch, \cite{Tappert:1990p10302}

% section klassifizierung (end)

Argumente für KNN

-> gleicht Schreibstile aus wenn genug Muster

% chapter erkennung_handgeschriebener_symbole (end)

%!TEX root = /Users/daniel/Documents/thesis/thesis.tex
\chapter{Benchmarks} % (fold)
\label{cha:benchmarks}

\TODO Verwendete Technik: Immer einen Parameter optimieren und den Rest fest lassen.

\section{Kleiner Datensatz}
\label{sec:kleiner_datensatz}

In diesem Abschnitt werden mithilfe eines Teils der verfügbaren Trainingsdaten die Verfahren auf ihre Güte überprüft und die Parameter optimiert. Der kleine Datensatz besteht aus 100 zufällig aus der Datenbank ausgewählten Symbolen.
Es wurden 75 Muster pro Symbolklassen zufällig ausgewählt wovon 50 für das Training und 50 als Testmuster verwendet wurden. Dies ergibt 2500 Tests.

Folgende Symbole werden verwendet:

\begin{quote}
$\$$,
$\{$,
$\copyright$,
$\}$,
$\S$,
$\&$,
$\#$,
$\%$,
$\checkmark$,
\aa,
\AA,
\ae,
\DH,
\DJ,
$\EUR$,
$\cup$,
$\oplus$,
$\times$,
$\ast$,
$\otimes$,
$\pm$,
$\cap$,
$\vee$,
$\cdot$,
$\odot$,
$\wedge$,
$\circ$,
$\bigotimes$,
$\prod$,
$\sum$,
$\bigodot$,
$\int$,
$\oint$,
$\approx$,
$\equiv$,
$\perp$,
$\cong$,
$\propto$,
$\vdash$,
$\sim$,
$\simeq$,
$\therefore$,
$\because$,
$\subseteq$,
$\geq$,
$\leq$,
$\ll$,
$\neq$,
$\lesssim$,
$\gtrsim$,
$\triangleq$,
$\Rightarrow$,
$\rightarrow$,
$\Leftrightarrow$,
$\mapsto$,
$\alpha$,
$\theta$,
$\tau$,
$\beta$,
$\vartheta$,
$\pi$,
$\gamma$,
$\phi$,
$\delta$,
$\rho$,
$\varphi$,
$\epsilon$,
$\lambda$,
$\chi$,
$\varepsilon$,
$\mu$,
$\sigma$,
$\psi$,
$\zeta$,
$\nu$,
$\omega$,
$\eta$,
$\xi$,
$\Gamma$,
$\Lambda$,
$\Sigma$,
$\Psi$,
$\Delta$,
$\Xi$,
$\Omega$,
$\Theta$,
$\Pi$,
$\Phi$,
$\bot$,
$\forall$,
$\ell$,
$\hbar$,
$\in$,
$\not\in$,
$\partial$,
$\exists$,
$[$,
$/$,
$\aleph$,
$\infty$
\end{quote}

Davon sind manche, wie $\odot$ (\verb!\odot!) und $\bigodot$ (\verb!\bigodot!) oder $\sum$(\verb!\sum!) und $\Sigma$~(\verb!\Sigma!) eigentlich dieselben Symbole. Es wurden aber keine Maßnahmen ergriffen, um solche Zweideutigkeiten aufzuheben, da durch die Interaktivität der Anwendung ohnehin in erste Linie eine hohe Top 5-Erkennungsrate wichtig ist.

\subsection{DTW-Variante}
\label{sub:gierig_oder_nicht}

Der erste Benchmark testet DTW in seiner klassischen Form gegen die Greedy-Approximation GDTW. Dabei wurde erst einmal $C=50$, $n=25$ (optimal nach \citet{Golubitsky:2009p1842}) und das Abstandsmaß $\delta$ als die euklidische Metrik festgelegt. Abbildung \ref{chart:dtw-vs-gdtw} illustriert die Ergebnisse. Es zeigt sich, dass GDTW kaum schlechter abschneidet, als DTW und dabei, wie aus Abbildung \ref{chart:dtw-vs-gdtw-ms} ersichtlich ist, wesentlich schneller ist. Dies bestätigt die Ergebnisse von \citet{MacLean:2010p9970}, die ebenfalls eine Greedy-Variante von DTW verwenden.

\begin{figure}[htbp]
  \begin{center}
    \includegraphics[width=\textwidth]{charts/dtw-vs-gdtw.pdf}
  \end{center}
  \caption{Greedy DTW gegen klassisches DTW}
  \label{chart:dtw-vs-gdtw}
\end{figure}

\begin{figure}[htbp]
  \begin{center}
    \includegraphics[width=.75\textwidth]{charts/dtw-vs-gdtw-ms.pdf}
  \end{center}
  \caption{Greedy DTW gegen klassisches DTW - Laufzeit}
  \label{chart:dtw-vs-gdtw-ms}
\end{figure}

Auf Basis dieser Daten fällt es leicht die Auswahl des Verfahrens für Detexify auf GDTW festzulegen und im Folgenden die Betrachtungen auf dieses zu beschränken.

% chapter benchmarks (end)

\subsection{Inneres Abstandsmaß} % (fold)
\label{sub:inneres_abstandsmaß}

Abbildung \ref{chart:dtw-vs-gdtw} zeigt neben den Erkennungsraten von DTW und GDTW mit euklidischer Metrik als Abstandsmaß auch noch die Erkennungsraten von GDTW mit der Manhattan-Metrik als Abstandsmaß, die als

\[ d(x,y) = \sum_i a_i - b_i \]

definiert ist. Dieses Abstandsmaß ist nicht nur günstiger in der Berechnung, sondern liefert auch noch eine leichte Verbesserung der Top 5-Erkennungsrate um 0,6\%. Die folgenden Benchmarks wurden daher alle mit Manhattan-Metrik als inneres Abstandsmaß $\delta$ durchgeführt.

\subsection{Anzahl Trainingsmuster} % (fold)
\label{sub:anzahl_trainingsmuster}

Die Anzahl der verwendeten Trainingsmuster $C$ hat natürlich einen Einfluss auf die Erkennungsraten.

\begin{figure}[htbp]
  \begin{center}
    \includegraphics[width=.75\textwidth]{charts/samples.pdf}
  \end{center}
  \caption{Einfluss der Anzahl der Trainingsmuster $C$}
  \label{chart:samples}
\end{figure}

Abbildung \ref{chart:samples} zeigt, dass, anders als \citet{Golubitsky:2009p1842} für ihre Tests beschreiben, die Erkennungsraten (insb. Top 1) für $C > 25$ noch deutlich steigen. Dies liegt vermutlich an einer höheren Variation in meinen Trainingsdaten, so dass sich eine größere Datenbasis auszahlt. Die Wahl von $C=50$ wurde jedoch aus beibehalten und nicht noch weiter erhöht, denn $C$ wirkt sich linear auf die Laufzeit des Erkennungsvorgangs aus.
% subsection anzahl_klassen (end)

\subsection{Anzahl Punkte pro Strich} % (fold)
\label{sub:anzahl_punkte_pro_strich}

Die Anzahl der Punkte pro Strich $n$ hat ebenso sowohl einen Einfluss auf die Erkennungsrate als auch auf die Laufzeit. Abbildung \ref{chart:points} zeigt dass ab $n=10$ gerade die Top 1-Erkennungsrate mit sinkendem $n$ stark abnimmt. Die Top 5-Erkennungsrate bleibt jedoch recht stabil. Für die weiteren Tests wurde $n=10$ als optimal festgelegt, da dies nur geringe Einbußen in der Erkennung bei einer gut Laufzeit liefert.

\begin{figure}[htbp]
  \begin{center}
    \includegraphics[width=.75\textwidth]{charts/points.pdf}
  \end{center}
  \caption{Einfluss der Anzahl der Punkte pro Strich $n$}
  \label{chart:points}
\end{figure}

% subsection anzahl_punkte_pro_strich (end)

\subsection{Dominante Punkte} % (fold)
\label{sub:dominante_punkte}

Alle bisherigen Tests wurden ohne die Filterung von Punkten mit geringer Krümmung vorgenommen. Abbildung \ref{chart:degree} zeigt einerseits, dass ...\TODO

\begin{figure}[htbp]
  \begin{center}
    \includegraphics[width=.75\textwidth]{charts/degree.pdf}
  \end{center}
  \caption{Einfluss des Winkels $\alpha$}
  \label{chart:degree}
\end{figure}

% subsection dominante_punkte (end)

\section{Großer Datensatz}
\label{sec:grosser_datensatz}

\TODO schreiben

627 Symbole (die mit mind 15 Trainingsmustern) pro Klasse max 100 Trainingsmuster davon ein drittel als Testmuster und der Rest Training
13411 Testmuster
27460 Trainingsmuster

\begin{verbatim}
  Flop 10:
  \ddag (latex2e, OT1) - 0.0% top 1
  \dotso (amsmath, OT1) - 0.0% top 1
  \Updelta (upgreek, OT1) - 0.0% top 1
  \ldotp (latex2e, OT1) - 0.0% top 1
  \dotsi (amsmath, OT1) - 0.0% top 1
  \Rbag (stmaryrd, OT1) - 0.0% top 1
  \dots (latex2e, OT1) - 0.0% top 1
  \Uppi (upgreek, OT1) - 0.0% top 1
  \Gemini (marvosym, OT1) - 0.0% top 1
  \therefore (amssymb, OT1) - 0.0% top 1
  Top 10:
  \cong (latex2e, OT1) - 93.93939393939394% top 1
  \not\equiv (latex2e, OT1) - 93.93939393939394% top 1
  \& (latex2e, OT1) - 93.93939393939394% top 1
  \supset (latex2e, OT1) - 93.93939393939394% top 1
  \asymp (latex2e, OT1) - 90.9090909090909% top 1
  \approxeq (amssymb, OT1) - 90.9090909090909% top 1
  \neq (amssymb, OT1) - 90.9090909090909% top 1
  \gtrsim (amssymb, OT1) - 90.9090909090909% top 1
  \rightharpoonup (latex2e, OT1) - 90.9090909090909% top 1
  \circledR (amssymb, OT1) - 90.9090909090909% top 1
  Global stats:
  Top 1: 52.214765100671144%
  Top 2: 69.9179716629381%
  Top 3: 77.59880686055183%
  Top 4: 81.89410887397464%
  Top 5: 84.53392990305743%
  Top 6: 86.14466815809098%
  Top 7: 87.44220730797912%
  Top 8: 88.25503355704699%
  Top 9: 88.91126025354214%
  Top 10: 89.63460104399702%
  Overall 13410 Tests for 627 Symbols in 11654.247962 secs needing 0.8690066335098053 secs per test.
\end{verbatim}

\TODO Darauf eingegen, dass gerade die schlecht erkannten Symbole sehr ähnlich zu anderen die häufiger trainiert wurden sind z.B. \verb!\Rbag! ($\Rbag$) und \verb!\int! ($\int$)...
%!TEX root = /Users/daniel/Documents/thesis/thesis.tex
\chapter{Zusammenfassung und Ausblick}
\label{cha:ausblick}

\section{Zusammenfassung}

\TODO Welche Verfahren werden denn nun eingesetzt? Parameterwahlen? Das müssen die Benchmarks zeigen\dots

Nutze GDTW mit $\delta=Euklid$, $C=50$, $n=10$ und $\alpha=15\degree$

\section{Ausblick}

\TODO Wie geht es weiter?

\begin{itemize}
  \item Vergleich weiterer innerer Maße
  \item Kombination von Klassifizierern
  \item Version, die ohne Web auskommt
  \item Parallelisierung durch mehrere Backends und LoadBalancing
  \item social training - Anreizsystem für gutes Training und Löschen schlechter Trainingsdaten?
  \item \dots
\end{itemize}

\appendix
%!TEX root = /Users/daniel/Documents/thesis/thesis.tex
\chapter{Nutzungsstatistiken} % (fold)
\label{cha:nutzungsstatistiken}

Dieses Kapitel enthält einige Graphen, die Aufschluss über die Benutzer von Detexify geben können. Die Daten stammen aus dem Zeitraum vom 22.8.2010 bis zum 21.9.2010. Die Nutzungsdaten werden permanent mithilfe von Google Analytics \cite{analytics} gesammelt. Ich gebe keine Interpretation der Daten an. Dem Leser steht frei seine eigenen Schlüsse zu ziehen.

\begin{figure}[htbp]
  \begin{center}
    \includegraphics[width=.75\textwidth]{usage/users.png}
  \end{center}
  \caption{Besucherzahlen innerhalb eines Monats. Es ist klar ein Wochenrhythmus erkennbar.}
\end{figure}

\begin{figure}[htbp]
  \begin{center}
    \includegraphics[width=.8\textwidth]{usage/os.pdf}
  \end{center}
  \caption{Betriebssysteme der Nutzer}
\end{figure}

\begin{figure}[htbp]
  \begin{center}
    \includegraphics[width=.8\textwidth]{usage/browser.pdf}
  \end{center}
  \caption{Webbrowser der Nutzer}
\end{figure}

\begin{figure}[htbp]
  \begin{center}
    \includegraphics[width=\textwidth]{usage/country.pdf}
  \end{center}
  \caption{Länder}
\end{figure}

\begin{figure}[htbp]
  \begin{center}
    \includegraphics[width=.75\textwidth]{usage/city.pdf}
  \end{center}
  \caption{Städte}
\end{figure}

\begin{figure}[htbp]
  \begin{center}
    \includegraphics[width=\textwidth]{usage/countrymap.png}
  \end{center}
  \caption{Karte der Zugriffe: Länder}
\end{figure}

\begin{figure}[htbp]
  \begin{center}
    \includegraphics[width=\textwidth]{usage/citymap.png}
  \end{center}
  \caption{Karte der Zugriffe: Städte}
\end{figure}


% chapter nutzungsstatistiken (end)

%!TEX root = /Users/daniel/Documents/thesis/thesis.tex
\chapter{Abkürzungen}

% Acronyme
\begin{acronym}[YTM]
\setlength{\itemsep}{-\parsep}
\acro{WYSIWYG}{What You See Is What You Get}
\acro{HTML}{Hyper Text Markup Language}
\acro{betai}[$\beta_i$]{Regressionskoeffizient}
\end{acronym}
%!TEX root = /Users/daniel/Documents/thesis/thesis.tex
\chapter{Benutzerhandbuch} % (fold)
\label{man:benutzerhandbuch}

\section{Symbolssuche} % (fold)
\label{man:symbolsuche}

Lorem ipsum dolor sit amet, consectetur adipisicing elit, sed do eiusmod tempor incididunt ut labore et dolore magna aliqua. Ut enim ad minim veniam, quis nostrud exercitation ullamco laboris nisi ut aliquip ex ea commodo consequat. Duis aute irure dolor in reprehenderit in voluptate velit esse cillum dolore eu fugiat nulla pariatur. Excepteur sint occaecat cupidatat non proident, sunt in culpa qui officia deserunt mollit anim id est laborum.

% section ein_symbol_suchen (end)

\section{Symbolstabelle} % (fold)
\label{man:symboltabelle}

Lorem ipsum dolor sit amet, consectetur adipisicing elit, sed do eiusmod tempor incididunt ut labore et dolore magna aliqua. Ut enim ad minim veniam, quis nostrud exercitation ullamco laboris nisi ut aliquip ex ea commodo consequat. Duis aute irure dolor in reprehenderit in voluptate velit esse cillum dolore eu fugiat nulla pariatur. Excepteur sint occaecat cupidatat non proident, sunt in culpa qui officia deserunt mollit anim id est laborum.

\subsection{Ein Symbol Trainieren} % (fold)
\label{man:training}

Lorem ipsum dolor sit amet, consectetur adipisicing elit, sed do eiusmod tempor incididunt ut labore et dolore magna aliqua. Ut enim ad minim veniam, quis nostrud exercitation ullamco laboris nisi ut aliquip ex ea commodo consequat. Duis aute irure dolor in reprehenderit in voluptate velit esse cillum dolore eu fugiat nulla pariatur. Excepteur sint occaecat cupidatat non proident, sunt in culpa qui officia deserunt mollit anim id est laborum.

% subsection training (end)

% section ein_symbol_trainieren (end)

% chapter benutzerhandbuch (end)
%!TEX root = /Users/daniel/Documents/thesis/thesis.tex
\chapter{Webtechnologien} % (fold)
\label{cha:webtechnologien}

\section{HTTP} % (fold)
\label{sec:http}

Lorem ipsum dolor sit amet, consectetur adipisicing elit, sed do eiusmod tempor incididunt ut labore et dolore magna aliqua. Ut enim ad minim veniam, quis nostrud exercitation ullamco laboris nisi ut aliquip ex ea commodo consequat. Duis aute irure dolor in reprehenderit in voluptate velit esse cillum dolore eu fugiat nulla pariatur. Excepteur sint occaecat cupidatat non proident, sunt in culpa qui officia deserunt mollit anim id est laborum.

% section http (end)

\section{AJAX} % (fold)
\label{sec:ajax}

Lorem ipsum dolor sit amet, consectetur adipisicing elit, sed do eiusmod tempor incididunt ut labore et dolore magna aliqua. Ut enim ad minim veniam, quis nostrud exercitation ullamco laboris nisi ut aliquip ex ea commodo consequat. Duis aute irure dolor in reprehenderit in voluptate velit esse cillum dolore eu fugiat nulla pariatur. Excepteur sint occaecat cupidatat non proident, sunt in culpa qui officia deserunt mollit anim id est laborum.

% section ajax (end)

\section{REST} % (fold)
\label{sec:rest}

Lorem ipsum dolor sit amet, consectetur adipisicing elit, sed do eiusmod tempor incididunt ut labore et dolore magna aliqua. Ut enim ad minim veniam, quis nostrud exercitation ullamco laboris nisi ut aliquip ex ea commodo consequat. Duis aute irure dolor in reprehenderit in voluptate velit esse cillum dolore eu fugiat nulla pariatur. Excepteur sint occaecat cupidatat non proident, sunt in culpa qui officia deserunt mollit anim id est laborum.

% section rest (end)

\section{JSON} % (fold)
\label{sec:json}

Lorem ipsum dolor sit amet, consectetur adipisicing elit, sed do eiusmod tempor incididunt ut labore et dolore magna aliqua. Ut enim ad minim veniam, quis nostrud exercitation ullamco laboris nisi ut aliquip ex ea commodo consequat. Duis aute irure dolor in reprehenderit in voluptate velit esse cillum dolore eu fugiat nulla pariatur. Excepteur sint occaecat cupidatat non proident, sunt in culpa qui officia deserunt mollit anim id est laborum.

% section json (end)

\section{SVG} % (fold)
\label{sec:svg}

Lorem ipsum dolor sit amet, consectetur adipisicing elit, sed do eiusmod tempor incididunt ut labore et dolore magna aliqua. Ut enim ad minim veniam, quis nostrud exercitation ullamco laboris nisi ut aliquip ex ea commodo consequat. Duis aute irure dolor in reprehenderit in voluptate velit esse cillum dolore eu fugiat nulla pariatur. Excepteur sint occaecat cupidatat non proident, sunt in culpa qui officia deserunt mollit anim id est laborum.

% section svg (end)

% chapter webtechnologien (end)

\chapter{Frameworks} % (fold)
\label{cha:frameworks}

\TODO brauche ich sowas auch?

\begin{itemize}
  \item Raphael
  \item jQuery
  \item Mustache
\end{itemize}

% chapter frameworks (end)
\chapter{Programmquellcode}

\begin{tiny}

% Mit verbatiminput bleibt das Layout eines Listings erhalten.\\

\verbatiminput{progam.hs}

\end{tiny}

\backmatter
%!TEX root = /Users/daniel/Documents/thesis/thesis.tex
%\chapter{Selbstständigkeitserklärung}

% \setlength{\parindent}{0mm}
% I hereby certify that this diploma thesis has been composed by myself, and describes my own work, unless otherwise acknowledged in the text. All references and verbatim extracts have been quoted, and all sources of information have been specifically acknowledged. It has not been accepted in any previous application for a degree.\\
% 
% \vspace{20mm}
% 
% Münster, \today\\
% 
% \vspace{15mm}
% 
% Daniel Kirsch\\

%%%%%%%%%%%%%%%%%%%%%%%%%%%%%%%%%%%%%%%%%%%%%%%%%%%%55
% Selbstst"andigkeisterkl"arung
\thispagestyle{empty}
  % \enlargethispage{20mm}
\hskip 0mm
\vfill
\bigskip\noindent Ich erkl"are hiermit, dass ich die vorliegende Arbeit
selbstst"andig verfasst und keine anderen als die angegebenen Quellen und Hilfsmittel verwendet habe.\par
\bigskip\noindent M"unster, \today\par
\vskip 10mm
\hfill\hrulefill
%!TEX root = /Users/daniel/Documents/thesis/thesis.tex
\chapter{Danksagungen}

% TODO
I want to thank the academy. And: Lorem ipsum dolor sit amet, consectetur adipisicing elit, sed do eiusmod tempor incididunt ut labore et dolore magna aliqua. Ut enim ad minim veniam, quis nostrud exercitation ullamco laboris nisi ut aliquip ex ea commodo consequat. Duis aute irure dolor in reprehenderit in voluptate velit esse cillum dolore eu fugiat nulla pariatur. Excepteur sint occaecat cupidatat non proident, sunt in culpa qui officia deserunt mollit anim id est laborum.

Korrekturleser Matthias Redecker, Wiebke Heep, Philipp Kühl

Philipp Kühl für die Idee

Xiaoyi Jiang für die kompetente Betreuung und gute Hinweise in der Anfangsphase des Projekts

Zweitag GmbH für das schöne Büro

Mojca Miklavec der auf der tug-summer-of-code Mailing-Liste wertvolles Feedback gegeben hat, als ich die ersten Alphaversionen von Detexify online hatte

Scott Pakin für die Comprehensive Latex Symbol List und der handwriting-based symbol search for Google Summer of Code vorgeschlagen hatte schon bevor ich mit dem Projekt begonnen hatte und hilfreichen Hinweisen in der tug-summer-of-code Mailingliste.

den Nutzern von Detexify, insb. denen, die Trainingsmustern gespendet haben

den Künstlern, die für \ref{cha:kunst} verantwortlich sind

den Spendern, die die Hostingkosten für mich bei +-0 gehalten haben
%!TEX root = /Users/daniel/Documents/thesis/thesis.tex
\bibliographystyle{alphadin}
\cleardoublepage
\phantomsection
\addcontentsline{toc}{chapter}{Literaturverzeichnis}
\bibliography{bibliography}

\end{document}

