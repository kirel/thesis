%!TEX TS-program = xelatex
%!TEX encoding = UTF-8 Unicode
\documentclass[12pt, a4paper]{book}
\usepackage{amssymb} % Mathe
\usepackage{amsmath} % Mathe
\usepackage{verbatim}

% Math Fonts
\usepackage[mathcal]{euler}
\usepackage[cm-default]{fontspec}
% FONTS
\usepackage{xunicode}
\usepackage{xltxtra}
\defaultfontfeatures{Mapping=tex-text} % converts LaTeX specials (``quotes'' --- dashes etc.) to unicode
\setromanfont [Ligatures={Common}, Numbers={OldStyle}]{Adobe Caslon Pro}
\setmonofont[Scale=0.8]{Monaco} 
\setsansfont[Scale=0.9]{Optima Regular} 
% ---- CUSTOM AMPERSAND
\newcommand{\amper}{{\fontspec[Scale=.95]{Adobe Caslon Pro}\selectfont\itshape\&}}

% HEADINGS
\pagestyle{headings}
% \usepackage{sectsty} 
% \usepackage[normalem]{ulem} 
% \sectionfont{\sffamily\mdseries\large} 
% \subsectionfont{\rmfamily\mdseries\scshape\normalsize} 
% \subsubsectionfont{\rmfamily\bfseries\upshape\normalsize} 

% DOCUMENT LAYOUT
\setlength{\paperwidth}{21cm}
\setlength{\paperheight}{29.5cm}
\setlength{\parindent}{6mm}
\setlength{\hoffset}{-1in}
\setlength{\voffset}{-1in}
\setlength{\oddsidemargin}{2.7cm}
\setlength{\evensidemargin}{2.2cm}
\setlength{\textwidth}{15.5cm}
\setlength{\textheight}{22.8cm}
\setlength{\topmargin}{2.5cm}
\setlength{\headheight}{0.5cm}
\setlength{\headsep}{1cm}
\setlength{\footskip}{1.4cm}

% PDF SETUP
% ---- FILL IN HERE THE DOC TITLE AND AUTHOR
\usepackage[xetex, bookmarks, colorlinks, breaklinks, pdftitle={LaTeX-Erkennung},pdfauthor={Daniel Kirsch}]{hyperref}  
\hypersetup{linkcolor=blue,citecolor=blue,filecolor=black,urlcolor=blue} 

% Stuff
\usepackage[ngerman]{babel}
\usepackage{indentfirst}
\usepackage[printonlyused]{acronym}

% \pagestyle{headings}
% 

\begin{document}

\frontmatter

\title{Detexify: Erkennung handgemalter \LaTeX-Symbole}

\author{Daniel Kirsch}

\date{\vspace{10mm} Diplomarbeit\\ \vspace{3mm} \today \\
\vspace{20mm}
% Thema gestellt von\\ \vspace{3mm} Prof. Dr. W.-M. Lippe\\
\vspace{3mm} Westfälische Wilhelms-Universität Münster\\Institut für Informatik}

\maketitle


\rule{0mm}{1mm}
\newpage
\rule{0mm}{1mm}
\newpage

% Fürs Prüfungsamt ein paar Seiten frei.

\tableofcontents

\chapter{Vorwort}

\section*{Wie es zu dieser Diplomarbeit kam}

Eines Tages saß ich mit einem Kommilitonen zusammen in der Mensa. Wir unterhielten uns über Ideen im Allgemeinen und welche es sich lohne umzusetzen. Wir hatten viele Ideen und meistens keine Zeit sie zu verwirklichen. Tatsächlich existierte sogar eine Liste mit einigen davon - ein schlichtes Blatt Papier, auf dem sich über die Zeit einiges angesammelt hatte und wir nannten es den Ideenfriedhof.

Am besagten Tag ging es vor allem darum, wie man auf {\em gute} Ideen käme. Denn auf dem Ideenfriedhof ruhte Sinnvolles und weniger Sinnvolles. Ich äußerte die Behauptung, dass Ideen, die eigene Probleme lösen, solche seinen, deren Umsetzung sich lohne, weil man in der Regel mit seinen Problemen nicht alleine dastehe. Somit löse man auch Probleme für andere.

Wir kamen auf andere Themen und da mein Kommilitone zu jener Zeit den Sketretärinnen seiner Arbeitsgruppe einen \LaTeX-Kurs gab, unterhielten wir uns auch darüber. Irgendwann bemerkte er: "`Da fällt mir etwas ein... Was fehlt ist eine LaTeX-Rückwärts-Suche. Es kommt ganz häufig vor, dass man zwar weiss, was man für ein Symbol haben will, aber nicht den Befehl kennt. Das Problem hatten meine Sekretärinnen jetzt auch schon ein paar Mal. Wenn man jetzt ein Programm hätte, in das man das Symbol malen könnte und das dann den richtigen Befehl ausspuckt..."'

Ich fand, dass das einen außergewöhnlich gute Idee war. Ich hatte das Problem zwar nicht selbst, aber ich würde es bald haben, wenn ich meine Diplomarbeit schreiben würde, denn ich würde sie in \LaTeX\ schreiben. Ich hatte allerdings zu dem Zeitpunkt noch keine Ahnung, was ich als Diplomarbeit schreiben würde. Ich hatte zu diesem Zeitpunkt auch keine Berührung mit Mustererkennung irgendeiner Art gehabt und keine Ahnung von \LaTeX. Ich fand aber das Problem so interessant, dass ich fest entschlossen war es anzugehen.

Einige Wochen später, nach vielem Lesen und Probieren, hatte ich eine benutzbare Version, implementiert als Webseite und bedienbar über einen Webbrowser. Seitdem wird die Anwendung täglich von fast 800 Besuchern genutzt. Erst als diese Arbeit getan war, ist mir klar geworden, dass sich das Thema doch ganz gut für eine Diplomarbeit eignet und zu meiner Freude war Prof. Jiang auch dieser Meinung. Zudem würde ich meine eigene Diplomarbeit bei der Erstellung meiner Diplomarbeit verwenden können. Wer kann das schon von sich behaupten?

Was aber ich eigentlich hier sagen will: Sie lesen nun im wesentlichen das Ergebnis eines \LaTeX-Kurses für Sekretärinnen der Arbeitsgruppe Topologie des Mathematischen Institutes der WWU Münster. Darum gilt den damaligen TeilnehmerInnen mein besonderer Dank.
%!TEX root = /Users/daniel/Documents/thesis/thesis.tex
\chapter{Motivation}

\section{\LaTeX}

Wenn es um das Verfassen wissenschaftlicher Texte geht, ist für viele \LaTeX\ die erste Wahl. Anders als bei einem \acs{WYSIWYG}-Editor kümmert man sich nicht um das Layout, um Abstände und Schriftgrößen sondern um die Semantik des Geschriebenen. Stattdessen zeichnet man die Kapitel, Abschnitte, Formeln usw. semantisch aus, eher wie in \acs{HTML}. Der Vorteil liegt klar auf der Hand. Inhalt und Präsentation sind klar getrennt, was zur Folge hat, dass man nicht vom Wesentlichen abgelenkt wird, wenn man an Texten und Formeln arbeitet, und dass das Aussehen des Dokuments zentral in der Präambel definiert wird.

\LaTeX\ hat jedoch auch seine Nachteile. Gerade Anfänger haben es aufgrund der flachen Lernkurve schwer. Da man \LaTeX\ in der Regel im Quelltext bearbeitet, als jeden Befehl von Hand schreibt. Muss man sich unheimlich viel merken. Das beginnt bei einfachen Befehlen wie \texttt{\textbackslash chapter}, deren Name sich geradezu aufdrängt, aber wenn es darum geht Mathematische Formeln wie $$\Gamma \left( x \right) = \int\limits_0^\infty  {s^{x - 1} e^{ - s} ds}$$ in das Dokument zu bringen, wird es schon schwieriger.

Dass man ein $\Gamma$ mit dem Befehl \texttt{\textbackslash Gamma} bekommt ist noch zu erahnen. Um $\infty$ zu bekommen hätte man auch \texttt{\textbackslash infinity} statt \texttt{\textbackslash infty} versuchen können und hätte bloß eine Fehler geerntet, aber bei Befehlen wie \texttt{\textbackslash leftrightsquigarrow} ($\leftrightsquigarrow$) hört jede Intuition auf. Es ist also offensichtlich, dass es einigen Raum für unterstützende Maßnahmen bei der Erstellung von \LaTeX-Dokumenten gibt. Das gilt insbesondere für mathematische Formeln in denen viele unterschiedliche Symbole vorkommen können.

\section{Die optimale Eingabemethode}

Überlegt man sich die natürlichste Art Text oder Mathematik zu notieren, so kommt man unweigerlich auf Stift und Papier. Während bei Texten die Eingabe über eine Computertastatur durchaus schneller sein kann, als das Schreiben mit einem Stift (TODO Belege!) ist spätestens bei Formeln klar, dass hier der Stift klar im Vorteil ist. Es können beliebige Formen, Zeichen und Symbole in beliebige räumliche Beziehung gebracht werden. Nichts liegt also näher, als diese Eingabeform in die digitale Welt übertragen zu wollen.

Als Eingabegerät bietet sich also ein Grafiktablett an. Ein entsprechender \LaTeX-Editor würde eine Fläche zur Verfügung stellen, auf die einfach geschrieben und skizziert wird. Die Kurven und Linien würden dann vom Editor in Text, Tabellen, Formeln und Diagramme überführt -- alles genau wie vom Benutzer erwartet. Leider sind wir noch nicht so weit.

Alleine der Bereich der mathematischen Formelerkennung ist noch nicht auf einem Level, auf dem man die verfügbaren Lösungen als benutzbar bezeichnen könnte. Ein Beispiel ist die kostenlose Software \href{http://www.inftyproject.org}{InftyEditor}. Einfache Formeln werden noch recht sicher erkannt, aber sobald die Komplexität der Formel steigt und vor allem sobald man Symbole braucht, die die Software gar nicht kennt, wird die Benutzung zum Frusterlebnis. Es wundert daher nicht, dass an diesem Problem viel aktuelle Forschung betrieben wird.

\TODO{Beispiele oder Zitate/Referenzen}

Das Problem teilt sich dabei in drei Teile auf. Das erste ist die Segmentierung der mathematischen Formel, bei der einzelne Symbole isoliert werden müssen. Als zweites müssen die einzelnen Symbole richtig erkannt werden. Schließlich müssen die Symbole über ihre räumliche Position in eine logische Beziehung zueinander gebracht werden. Diese in diesen Schritten gewonnenen Erkenntnisse können natürlich die Entscheidung in den jeweils anderen beeinflussen, indem man die Semantik einer Interpretation in der Erkennung mit einfließen lässt. \TODO Zitate? Das Problem ist also sehr komplex.

\TODO{Hier könnte ein wenig Forschungsübersicht gegeben werden}

\section{Suche nach Alternativen}

Die optimale Eingabemethode ist also (noch) nicht praktikabel. Um ganze Formeln zu erkennen haben wir noch nicht die optimalen Algorithmen gefunden. Eine Alternative ist, dem Computer nur einen Teil der Erkennung zu übertragen. Ein Beispiel hierfür ist \href{http://jequation.sourceforge.net/}{JEquation}. In diesem Programm wird dem Benutzer vorgegeben, wo der zu malen hat, damit die Struktur der Formel richtig erkannt wird. \TODO Bild? Es bleibt also die Aufgabe die einzelnen Symbole zu erkennen.



\href{http://lyx.org}{Lyx}


\section{Ein pragmatischer Ansatz}



\url{http://url.de}

\mainmatter
%!TEX root = /Users/daniel/Documents/thesis/thesis.tex
\chapter{Chapter}

\section{Section}

Lorem ipsum dolor sit amet, consectetur adipisicing elit, sed do eiusmod tempor incididunt ut labore et dolore magna aliqua. Ut enim ad minim veniam, quis nostrud exercitation ullamco laboris nisi ut aliquip ex ea commodo consequat. Duis aute irure dolor in reprehenderit in voluptate velit esse cillum dolore eu fugiat nulla pariatur. Excepteur sint occaecat cupidatat non proident, sunt in culpa qui officia deserunt mollit anim id est laborum.

\newpage

\backmatter

\appendix

%!TEX root = /Users/daniel/Documents/thesis/thesis.tex
\chapter{Zusammenfassung und Ausblick}

\TODO

\begin{itemize}
  \item Vergleich verschiedener innerer Maße
  \item Kombination von Klassifizierern
  \item Version, die ohne Web auskommt
  \item \dots
\end{itemize}
\chapter{Danksagungen}

% TODO
I want to thank the academy. And: Lorem ipsum dolor sit amet, consectetur adipisicing elit, sed do eiusmod tempor incididunt ut labore et dolore magna aliqua. Ut enim ad minim veniam, quis nostrud exercitation ullamco laboris nisi ut aliquip ex ea commodo consequat. Duis aute irure dolor in reprehenderit in voluptate velit esse cillum dolore eu fugiat nulla pariatur. Excepteur sint occaecat cupidatat non proident, sunt in culpa qui officia deserunt mollit anim id est laborum.

% TODO foo

%!TEX root = /Users/daniel/Documents/thesis/thesis.tex
\chapter{Abkürzungen}

% Acronyme
\begin{acronym}[YTM]
\setlength{\itemsep}{-\parsep}
\acro{API}{Application Programming Interface}
\acro{CAS}{Computer Algebra System}
\acro{WYSIWYG}{What You See Is What You Get}
\acro{WYSIWYM}{What You See Is What You Mean}
\acro{HTML}{Hyper Text Markup Language}
\acro{URL}{Uniform Resource Locator}
\acro{URI}{Uniform Resource Identifier}
\acro{HATEOAS}{Hypermedia as the Engine of Application State}
\acro{REST}{Representational State Transfer}
\acro{HTTP}{Hypertext Transfer Protocol}
\acro{JSON}{Javascript Object Notation}
\acro{SVG}{Scalable Vector Graphics}
\acro{betai}[$\beta_i$]{Regressionskoeffizient}
\acro{SVM}{Support Vector Machines}
\acro{DAGSVM}{directed acyclic graph-SVM}
\acro{WTA}{Winner-takes-all}
\acro{MWV}{Max-wins voting}
\acro{HMM}{Hidden Markov Model}
\acro{NN}{Neuronale Netze}
\acro{k-NN}{k-nearest neighbor}
\acro{DTW}{dynamic time warping}
\acro{GDTW}{greedy dynamic time warping}
\end{acronym}


% TODO...
\chapter{Programlisting}

\begin{tiny}

% Mit verbatiminput bleibt das Layout eines Listings erhalten.\\

\verbatiminput{progam.hs}

\end{tiny}

\begin{thebibliography}{90}

\bibitem{Autor}
Autor\\
Titel\\
Verlag\\

\end{thebibliography}

\chapter{Statement of authorship}

\setlength{\parindent}{0mm}
I hereby certify that this diploma thesis has been composed by myself, and describes my own work, unless otherwise acknowledged in the text. All references and verbatim extracts have been quoted, and all sources of information have been specifically acknowledged. It has not been accepted in any previous application for a degree.\\

\vspace{20mm}

Münster, \today\\

\vspace{15mm}

Daniel Kirsch\\

\end{document}

