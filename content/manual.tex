%!TEX root = /Users/daniel/Documents/thesis/thesis.tex
\chapter{Benutzerhandbuch} % (fold)
\label{man:benutzerhandbuch}

In diesem Kapitel wird die Benutzung von Detexify erläutert. Die Anwendung wird aufgerufen indem man Adresse \url{http://detexify.kirelabs.org} besucht.

\section{Symbolssuche} % (fold)
\label{man:symbolsuche}

Die erste Ansicht, nachdem man die Anwendung aufgerufen hat, ist die Symbolsuche. Der schwarz umrahmte quadratische Bereich mit dem Hinweis \textbf{\texttt{Draw here!}} ist die Zeichenfläche. Auf dieser kann ein Symbol gezeichnet werden und nach jedem abgeschlossenen Strich beginnt der Erkennungsvorgang\marginline{\textsf{\textbf{Erkennung}}}. Ein Indikator in der rechten unteren Ecke der Zeichenfläche weist auf einen aktiven Erkennungsvorgang hin. Währenddessen können weitere Striche auf die Zeichenfläche gezeichnet werden. Mit einen Klick auf \textbf{\texttt{clear}} in der unteren linken Ecke der Zeichenfläche kann der Erkennungsvorgang abgebrochen werden und der Inhalt der Zeichenfläche wird gelöscht.

Ist der Erkennungsvorgang abgeschlossen, so erscheint rechts neben der Zeichenfläche die Ergebnisliste\marginline{\textsf{\textbf{Ergebnisliste}}}. Diese enthält für jedes Ergebnis eine Grafik des Symbols, den Befehl und gegebenenfalls das Paket, das benötigt wird. Es werden zuerst nur die besten fünf Ergebnisse angezeigt. Am unteren Ende der Ergebnisliste kann mit einem Klick die vollständige Ergebnisliste angezeigt werden.

Nun kann noch mit einen Klick auf die Grafik eines Ergebnisses das entsprechende Symbol mit der aktuellen Zeichnung trainiert werden.\marginline{\textsf{\textbf{Training}}}

Oberhalb der Zeichenfläche befindet sich die Navigationsleiste.\marginline{\textsf{\textbf{Navigation}}} Über diese kann man zur Symboltabelle gelangen.

% section ein_symbol_suchen (end)

\section{Symbolstabelle} % (fold)
\label{man:symboltabelle}

Über den Besuch der Adresse \url{http://detexify.kirelabs.org/symbols.html} oder über die Navigationsleiste durch einen Klick auf \textbf{\texttt{symbols}}. Die Symboltabelle befindet sich auf der linken Seite der Ansicht und ähnelt der Ergebnisliste bei der Symbolsuche.

Am oberen Rand der Tabelle befinden sich Bedienelemente, die es ermöglichen die Symboltabelle nach Befehlen oder Paketen zu filtern\marginline{\textsf{\textbf{Filter}}}. Dazu muss der gewünschte Filter ausgewählt werden und eine beliebige Zeichenfolge, nach der gefiltert werden soll, in das vorhandene Textfeld eingegeben werden.

Klickt man auf einen Eintrag der Symboltabelle, so öffnet sich unterhalb dieses Eintrags eine Zeichenfläche. Auf diese kann dann das Symbol gezeichnet werden und mit einem Klick auf \textbf{\texttt{train}}\marginline{\textsf{\textbf{Training}}} am unteren linken Rand der Zeichenfläche wird das Symbol mit der Zeichnung trainiert. Die Zeichnung kann auch ohne das Symbol zu trainieren wieder gelöscht werden, indem daneben auf \textbf{\texttt{cancel}} geklickt wird.

Oberhalb der Filterkontrollen befindet sich wieder die Navigationsleiste.\marginline{\textsf{\textbf{Navigation}}} Über diese kann man durch einen Klick auf \textbf{\texttt{classify}} zurück zur Symbolsuche gelangen.

% subsection training (end)

% section ein_symbol_trainieren (end)

% chapter benutzerhandbuch (end)