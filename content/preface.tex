%!TEX root = /Users/daniel/Documents/thesis/thesis.tex
\chapter{Vorwort}

\section*{Wie es zu dieser Diplomarbeit kam...}

Eines Tages saß ich mit einem Kommilitonen zusammen in der Mensa. Wir unterhielten uns über Ideen im Allgemeinen und welche es sich lohne umzusetzen. Wir hatten viele Ideen und meistens keine Zeit sie zu verwirklichen. Tatsächlich existierte sogar eine Liste mit einigen davon - ein schlichtes Blatt Papier, auf dem sich über die Zeit einiges angesammelt hatte und wir nannten es den Ideenfriedhof.

Am besagten Tag ging es vor allem darum, wie man auf {\em gute} Ideen käme. Denn auf dem Ideenfriedhof ruhte Sinnvolles und weniger Sinnvolles. Ich äußerte die Behauptung, dass Ideen, die eigene Probleme lösen, solche seien, deren Umsetzung sich lohne, weil man in der Regel mit seinen Problemen nicht alleine dastehe. Somit löse man auch Probleme für andere.

Wir kamen auf andere Themen und da mein Kommilitone zu jener Zeit den Sekretärinnen des Mathematischen Instituts einen \LaTeX-Kurs gab, unterhielten wir uns auch darüber. Irgendwann bemerkte er: "`Da fällt mir etwas ein... Was fehlt ist eine \LaTeX-Rückwärts-Suche. Es kommt ganz häufig vor, dass man zwar weiß, was man für ein Symbol haben will, aber nicht den Befehl kennt. Das Problem hatten meine Sekretärinnen jetzt auch schon ein paar Mal. Wenn man jetzt ein Programm hätte, in das man das Symbol malen könnte und das dann den richtigen Befehl ausspuckt..."'

Ich fand, dass das eine außergewöhnlich gute Idee war. Ich hatte das Problem zwar nicht selbst, aber ich würde es bald haben, wenn ich meine Diplomarbeit schreiben würde, denn ich würde sie in \LaTeX\ schreiben. Ich hatte allerdings zu dem Zeitpunkt noch keine Ahnung, was ich als Diplomarbeit schreiben würde. Ich hatte zu diesem Zeitpunkt auch keine Berührung mit Mustererkennung irgendeiner Art gehabt und keine Ahnung von \LaTeX. Ich fand aber das Problem so interessant, dass ich fest entschlossen war es anzugehen.

Einige Wochen später, nach vielem Lesen und Probieren, hatte ich eine benutzbare Version, implementiert als Website und bedienbar über einen Browser. Seitdem wird die Anwendung täglich von fast 1000 Besuchern genutzt. Erst später wurde mir klar, dass sich das Thema gut für eine Diplomarbeit eignet und zu meiner Freude war Prof. Jiang auch dieser Meinung. Zudem würde ich meine eigene Diplomarbeit bei der Erstellung meiner Diplomarbeit verwenden können. Wer kann das schon von sich behaupten?

Was aber ich eigentlich hier sagen will: Sie lesen nun im wesentlichen das Ergebnis eines \LaTeX-Kurses für Sekretärinnen des Mathematischen Institutes der WWU Münster. Darum gilt den damaligen Teilnehmerinnen mein besonderer Dank.

\newpage
\pagestyle{empty}
\mbox{}\vfill
\begin{center}
\large
Für Julia Fiege, Yu-Mei Kao, Angela Odermann \amper~ Tamara Tietmeyer.
\normalsize
\end{center}
\vfill\mbox{}\newpage
\pagestyle{headings}