%!TEX root = /Users/daniel/Documents/thesis/thesis.tex
\chapter{Motivation}

\section{\LaTeX}

Wenn es um das Verfassen wissenschaftlicher Texte geht, ist für viele \LaTeX\ die erste Wahl. Anders als bei einem \ac{WYSIWYG}-Editor kümmert man sich nicht um das Layout, um Abstände und Schriftgrößen sondern um die Semantik des Geschriebenen. Man zeichnet die Kapitel, Abschnitte, Formeln usw. aus, eher wie in \ac{HTML} als in MS-Word. Der Vorteil liegt klar auf der Hand. Inhalt und Präsentation sind klar getrennt, was zur Folge hat, dass man nicht vom Wesentlichen abgelenkt wird, wenn man an Texten und Formeln arbeitet, und dass das Aussehen des Dokuments zentral in der Präambel definiert wird.

\LaTeX\ hat jedoch auch seine Nachteile. Gerade Anfänger haben es aufgrund der flachen Lernkurve schwer. Da man \LaTeX\ in der Regel im Quelltext bearbeitet, also jeden Befehl von Hand schreibt, muss man sich unheimlich viel merken. Das beginnt bei einfachen Befehlen wie \texttt{\textbackslash chapter}, deren Name sich geradezu aufdrängt, aber wenn es darum geht Mathematische Formeln wie $$\Gamma \left( x \right) = \int\limits_0^\infty  {s^{x - 1} e^{ - s} ds}$$ in das Dokument zu bringen, wird es schon schwieriger.

Dass man ein $\Gamma$ mit dem Befehl \texttt{\textbackslash Gamma} bekommt ist noch zu erahnen. Um $\infty$ zu bekommen hätte man auch \texttt{\textbackslash infinity} statt \texttt{\textbackslash infty} versuchen können und hätte bloß einen Fehler geerntet, aber bei Befehlen wie \texttt{\textbackslash leftrightsquigarrow} ($\leftrightsquigarrow$) hört jede Intuition auf. Es ist also offensichtlich, dass es einigen Raum für unterstützende Maßnahmen bei der Erstellung von \LaTeX-Dokumenten gibt. Das gilt insbesondere für mathematische Formeln in denen viele unterschiedliche Symbole vorkommen können.

\section{Die optimale Eingabemethode}

Überlegt man sich die natürlichste Art Text oder Mathematik zu notieren, so kommt man unweigerlich auf Stift und Papier. Während bei Texten die Eingabe über eine Computertastatur durchaus schneller sein kann als das Schreiben mit einem Stift (\TODO Belege!), ist spätestens bei Formeln klar, dass hier der Stift im Vorteil ist. Es können beliebige Formen, Zeichen und Symbole in beliebige räumliche Beziehung gebracht werden. Nichts liegt also näher, als diese Eingabeform in die digitale Welt übertragen zu wollen.

Als Eingabegerät bietet sich also ein Grafiktablett an. Ein entsprechender \LaTeX-Editor würde eine Fläche zur Verfügung stellen, auf die einfach geschrieben und skizziert wird. Die Kurven und Linien würden dann vom Editor in Text, Tabellen, Formeln und Diagramme überführt -- alles genau wie vom Benutzer erwartet. Leider sind wir noch nicht so weit.

Alleine der Bereich der mathematischen Formelerkennung ist noch nicht auf einem Level, auf dem man die verfügbaren Lösungen als benutzbar bezeichnen könnte. Ein Beispiel ist die kostenlose Software \href{http://www.inftyproject.org}{InftyEditor}. Einfache Formeln werden noch recht sicher erkannt, aber sobald die Komplexität der Formel steigt und vor allem sobald man Symbole braucht, die die Software gar nicht kennt, wird die Benutzung zum Frusterlebnis. Es wundert daher nicht, dass an diesem Problem viel aktuelle Forschung betrieben wird.

\TODO{Beispiele oder Zitate/Referenzen}

Das Problem teilt sich dabei in drei Teile auf. Das erste ist die Segmentierung der mathematischen Formel, bei der einzelne Symbole isoliert werden müssen. Als zweites müssen die einzelnen Symbole richtig erkannt werden. Schließlich müssen die Symbole über ihre räumliche Position in eine logische Beziehung zueinander gebracht werden. Die in diesen Schritten gewonnenen Erkenntnisse können natürlich die Entscheidung in den jeweils anderen beeinflussen, indem man die Semantik einer Interpretation in der Erkennung mit einfließen lässt. \TODO Zitate? Das Problem ist also sehr komplex.

\TODO{Hier könnte ein wenig Forschungsübersicht gegeben werden}

\section{Suche nach Alternativen}

Die optimale Eingabemethode ist also (noch) nicht praktikabel. Um ganze Formeln zuverlässig zu erkennen haben wir noch nicht die optimalen Algorithmen gefunden. Eine Alternative ist, dem Computer nur einen Teil der Erkennung zu übertragen. Ein Beispiel hierfür ist \href{http://jequation.sourceforge.net/}{JEquation}. In diesem Programm wird dem Benutzer vorgegeben, wo der zu malen hat, damit die Struktur der Formel richtig erkannt wird. Es bleibt also die Aufgabe die einzelnen Symbole zu erkennen. \TODO Bild? Und welche Kritik habe ich an diesem Ansatz?

Es gibt natürlich auch Ansätze, die ohne jede Form von Mustererkennung auskommen. Hier ist \href{http://lyx.org}{Lyx} ein Beispiel. Lyx ist ein \ac{WYSIWYM}-Editor, er funktioniert also ähnlich wie ein \ac{WYSIWYG}-Editor wie MS-Word. Dabei arbeitet man nicht direkt mit \LaTeX-Befehlen sondern hat einen graphischen Editor in dem man jedoch die Textabschnitte semantisch auszeichnet, statt den Schriftstil manuell vorzugeben. Für Mathematische Formeln enthält Lyx einen Formeleditor, der ähnlich dem funktioniert, was aus Office-Paketen wie MS-Word bekannt ist. Um Symbole einzufügen hat man einerseits die Möglichkeit direkt \LaTeX-Befehle (mit Auto-Vervollständigung) direkt einzugeben, oder das gesuchte Symbol aus mehreren Symboltabellen auszuwählen und per Klick einzufügen. Die erste der beiden Möglichkeiten setzt natürlich wieder voraus, dass der Author den Namen des Befehls kennt. Die zweite Methode hat den Nachteil, dass die Symboltabellen schnell unübersichtlich werden, wenn sie zu umfangreich sind. Es lässt sich also nur ein kleiner Teil der verfügbaren Symbole unterbringen, ohne den Nutzen zu kompromittieren.

\section{Ein pragmatischer Ansatz}

Viele \LaTeX-Benutzer fühlen sich aber durchaus wohl damit ihre Dokumente direkt im Quelltext zu bearbeiten. Texteditoren wie Vim, Emacs etc. erfreuen sich großer Beliebtheit zum erstellen von \LaTeX-Dokumenten. (\TODO Wie belege ich das? Internet?) Ich selbst ziehe den Texteditor TextMate trotz umfangreicher Recherchen in diesem Bereich jedem spezialisierten \LaTeX-Editor vor. Die Frage ist also, wie ein pragmatisches Werkzeug aussieht, dass einem typischen \LaTeX-Anwender die Arbeit an seinem Dokument insbesondere die Eingabe von mathematischen Formeln erleichtert.



\url{http://url.de}